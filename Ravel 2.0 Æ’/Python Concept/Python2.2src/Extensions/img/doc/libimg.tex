\chapter{Image file modules}

This section describes a number of modules that provide a more-or-less
unified interface to reading and writing images in various file
formats.
The philosophy used is that an in-core image will always be stored in
a format that is convenient to the application, and that conversion from/to
the external format is done upon reading or writing the image.

The incore image format defines all attributes of the graphical data:
type of pixel (eg. rgb or greyscale), number of bits, direction
(top-to-bottom or bottom-to-top), alignment rules, etc. All predefined
image formats are defined in the module \module{imgformat}.

Image file readers and writers usually support only a limited number
of image formats (one or two, usually), that correspond most closely
to the format the data has in the image file. In addition, there are
converters that can be stacked on the readers and writers that convert
the data to/from the internal format wanted. As an example: we want to
read a GIF image (which can only be read as 8-bit colormap data) but
we need 32bit RGB data. The code fragment
\begin{verbatim}
    tmp = imggif.reader('filename.gif')
    rdr = imgconvert.stackreader(imgformat.rgb, tmp)
\end{verbatim}
will allow us to use \var{rdr} to read RGB data.

To make life even easier, it is often unnecessary to use the
format-specific modules. The \module{img} module is a generalised
wrapper that will choose the correct reader or writer (based on magic
numbers in the file, or on filename extension). It will also stack the
needed converters to get the incore format you specify.

\section{General interface to image file modules}
\declaremodule{}{imagefile}
\modulesynopsis{General interface to image file modules.}
In general each image file module will export the following functions:

\begin{funcdesc}{reader}{file}
Create a reader for image file \var{file}, which can be a filename or
an open file object. The header will be
read, and the attributes set accordingly. An imagefile reader object
is returned.
\end{funcdesc}

\begin{funcdesc}{writer}{file}
Create a writer for image file \var{file}, which can be a filename or
an open file object. Before calling the
writers \method{write()} method you have to fill in the necessary
attributes (at least \var{width}, \var{height} and \var{format}, and
maybe \var{colormap} as well). An imagefile writer object is returned.
\end{funcdesc}

\subsection{Imagefile Objects}
\label{imagefile-objects}

The reader and writer objects have a number of common attributes and
methods. There may be more attributes for readers or writers for a
given file type, these are described in the relevant section.

\begin{memberdesc}[imagefile]{width}
The width of the image in pixels.
\end{memberdesc}
\begin{memberdesc}[imagefile]{height}
The height of the image in pixels.
\end{memberdesc}
\begin{memberdesc}[imagefile]{format_choices}
A tuple containing all image formats that this reader or writer can
directly support.
\end{memberdesc}
\begin{memberdesc}[imagefile]{format}
The image format that will be used for the incore data upon read or
write. The user can set this to any of the formats present in
\var{format_choices}.
\end{memberdesc}
\begin{memberdesc}[imagefile]{colormap}
For colormap-based file formats, this attribute contains a
\code{colormap} object representing the colormap data read from the
image file (or to be written to it).
\end{memberdesc}
\begin{memberdesc}[imagefile]{__dict__}
The full dictionary of attributes of the reader or writer. This can be
used to see which attributes the object supports.
\end{memberdesc}

\begin{methoddesc}[imagefile]{read}{}
Read the image data, returning a string. The \var{format} attribute
determines how the return value is structured.
\end{methoddesc}

\begin{methoddesc}[imagefile]{write}{img}
Write the image data from string \var{img}. \var{Img} is interpreted
according to the image format specified in \var{format}.
\end{methoddesc}

\section{\module{img} ---
         Interface for Reading and Writing Images.}
\declaremodule{}{img}
\modulesynopsis{Interface for Reading and Writing Images.}

This is the module of choice to use for applications where you simply
want to read an image in any format and get it incore in a format that
suits the application. The reader and writer objects use the interface
descriped above; the creation routines have a slightly different
interface, though. The creators raise the \var{unsupported_error}
exception when the conversion requested is not available.

\begin{funcdesc}{reader}{format, file\optional{, ignoreext}}
Create an imagefilereader that will read \code{file} and produce
internal image format \code{format}. If the file is specified by name
the image file type is deduced from the extension (unless the optional
\var{ignoreext} argument is non-zero). If this fails the 
first few bytes of the file are read to determine the type. A reader
for the correct type is then created. If the reader does not directly
support the requested
\code{format} a converter will be stacked. Alternatively, you can
supply \code{None} as \var{format}, and you will get the file reader
without any stacked converters.
\end{funcdesc}

\begin{funcdesc}{writer}{format, filename}
Create an imagefile writer that will write \var{filename} from an
image in internal format \var{format}. The image file writer to use
is determined by looking at the extension of the filename. If the
writer does not directly support the requested \var{format} a
converter will be stacked. Again, passing \var{None} will return the
file writer without any stacked converters.
\end{funcdesc}

\begin{funcdesc}{setquality}{high}
Some conversions can be done in high quality or low quality. This is
currently true for the 24-bit to 8-bit converters, that can use a
simple nearest-color search or a Floyd-Steinberg error
correction. High quality mode is on by default, but can take a rather
substantial amount of time, so it can be turned off by calling
\code{setquality(0)}. The function returns the old quality setting.

Setting the environment variable \var{SETDITHER} is equivalent to
calling \code{setquality(0)} upon importing the module.
\end{funcdesc}

\begin{funcdesc}{settrace}{on}
Calling \method{settrace} with a non-zero argument cause the converters
to print messages explaining how the conversion is done. Settrace
returns the old trace setting.
\end{funcdesc}

Note that \method{setquality} and \method{settrace} actually live in the
\module{imgformat} module but are re-exported by \module{img}.

\section{imgformat ---
	Predefined image formats.}
\declaremodule{extension}{imgformat}
\modulesynopsis{Predefined image formats.}

This module contains a number of predefined image formats, and allows
creation of new ones. You can check image formats for equality,
pass them around to other img routines and examine some of the
properties of the format. The formats all describe pictures stored in
row-major, left-to-right, top-to-bottom order, unless otherwise noted.

The module defines the following attributes and functions:

\begin{excdesc}{error}
the exception raised by all img modules. All img modules contain an
alias to it by the same name.
\end{excdesc}

\begin{datadesc}{colormap}
8-bit colormapped data as a byte array. Each
row is aligned on a 32-bit boundary. This is the format SGI GL
expects.
\end{datadesc}

\begin{datadesc}{colormap_b2t}
Idem, but ordered bottom-to-top.
\end{datadesc}

\begin{datadesc}{grey}
8-bit greyscale data as a byte array. Rows are
aligned on 32-bit boundaries.
\end{datadesc}

\begin{datadesc}{grey_b2t}
Idem, but ordered bottom-to-top.
\end{datadesc}

\begin{datadesc}{macrgb}
Macintosh 24-bit RGB format (which has the colors ordered different
from X or SGI format).
\end{datadesc}

\begin{datadesc}{macrgb16}
Macintosh 16-bit (15, actually) RGB format.
\end{datadesc}

\begin{datadesc}{pbmbitmap}
A bitmap with each pixel (bit) stored in a separate byte, in the
low-order bit.
\end{datadesc}

\begin{datadesc}{rgb}
24-bit RGB data or 32-bit RGBA data stored as a long array.
The red data is stored in the LSB, the (optional) alpha
data in the MSB. This is the SGI GL format.
\end{datadesc}

\begin{datadesc}{rgb_b2t}
Idem, ordered bottom-to-top.
\end{datadesc}

\begin{datadesc}{rgb8}
8-bit 3:3:2 RGB data stored as a byte array. Rows are
aligned on 32-bit boundaries. Each pixel has the format RRRBBGGG. This
is a format supported by some SGI graphics and videoboards.
\end{datadesc}

\begin{datadesc}{rgb8_b2t}
Idem, ordered bottom-to-top.
\end{datadesc}

\begin{datadesc}{xgrey}
Identical to \var{grey} but with 8-bit alignment.
\end{datadesc}
\begin{datadesc}{xrgb8}
Identical to \var{rgb8} but with 8-bit alignment.
\end{datadesc}
\begin{datadesc}{xcolormap}
Identical to \var{colormap}, but with 8-bit alignment.
\end{datadesc}

%** repeat the following for each function:
\begin{funcdesc}{new}{name, descrtext, properties}
Create a new image format object. \var{Name} is an identifier under
which the new format is known, \var{descrtext} argument is a short
textual description of the format, for instance \code{'SGI 8bit grey
top-to-bottom'}, \var{properties} is a dictionary of properties of the
format, see below.
\end{funcdesc}

Note that defining a new format is useful only if you at least also
supply a converter-function to/from some known format to the
\module{imgconvert} module.

The property-dictionary \var{descr} of a format can currently contain
the following predefined properties:
\code{'size'}, the size in bits of a pixel,
\code{'align'}, the alignment of rows, again in bits,
\code{'b2t'}, which is non-zero for bottom-to-top orientation,
\code{'type'}, the type of pixels (\code{'rgb'}, \code{'grey'},
\code{'mapped'} or \code{'packed'}),
\code{'comp'}, a list of 2-tuples giving bit-offset (counting from the
right end of the pixel) and bitcount per component (grey and mapped
formats have one component, rgb has 3 (R, G and B) or 4 (R, G, B and
A),
\code{'shift'} (\code{packed} only) the bit-position of the low-order
bit of the first pixel in the byte, counting from the left, and
\code{'step'} (\code{packed} only) how much to add to \code{shift} to
get the bit-position of the low-order bit of the next pixel.

Note that two formats are only equal if they compare equal: equality
of the property dictionaries is not enough (there may be hidden
properties of a format that are not easily described).

\section{imgcolormap ---
	Support for colormap objects.}
\declaremodule{extension}{imgcolormap}
\modulesynopsis{Support for colormap objects.}

This module contains support for colormap objects. A colormap object
is created by imagefile readers upon reading a colormap-based file
format (and expected by writers of those). A colormap object can also
be examined, changed and used to convert colormap data to rgb
and vice versa. Normally, this module will not be used directly by the
user, \module{img} will provide the calls to it, when needed.

\begin{funcdesc}{new}{str}
Create a new colormap object. The object mapping is initialized from
the string \var{str}, which should contain four bytes for each entry
(red LSB, MSB ignored, the \var{rgb} pixel format).
\end{funcdesc}

\begin{funcdesc}{new3}{redstr, greenstr, bluestr}
Create a new colormap object. The object mapping is initialized from
the three strings \var{redstr}, \var{greenstr} and \var{bluestr},
which should have the same length.
\end{funcdesc}

\begin{funcdesc}{fromimage}{data, width, height, format\optional{, maplen}}
Create the ``best'' colormap for a given image. Currently, the only
image format supported is \var{rgb}. The optional \var{maplen}
parameter tells how many entries the colormap should have, the default
is 256.

If the source image contains too many different colors (more that 64K,
currently) \var{fromimage} will fail and you should cluster your
colors. The \var{imgconvert} routines handle this case by using
\code{imgop.shuffle} to use components with fewer bits.
\end{funcdesc}

\subsection{Colormap Objects}
\label{colormap-objects}

A colormap object cannot change size after creation, so if you want to
create an empty 256-entry colormap which you can fill later you have
to call \code{imgcolormap.new(chr(0)*4*256)}.

Colormap objects partially behave as a sequence of \code{(red, green,
blue)} tuples, in that indexing them works (also for assignment) and
that the \code{len()} function works. They do not support slicing and
concatenation, though.

They also provide the following methods:

\begin{methoddesc}{map}{data, width, srcfmt, dstfmt}
Attempts to convert the image \var{data} of image format \var{srcfmt}
to an image in format \var{dstfmt}. The \var{width} argument gives the
width of the image. Currently the only supported conversions are from
\var{colormap} or \var{xcolormap} to \var{rgb} and from \var{colormap_b2t} to
\var{rgb_b2t}.
\end{methoddesc}

\begin{methoddesc}[colormap]{map8}{data}
Convert an 8-bit image to another 8-bit image (using the ``red'' entry
in the colormap).
\end{methoddesc}

\begin{methoddesc}[colormap]{getmacmapdata}{}
Return the colormap as a string in Macintosh-format: 4 2-byte integers
per entry, specifying index, red, green and blue values.
\end{methoddesc}

\begin{methoddesc}[colormap]{dither}{data, width, height, format\optional{, floyd}}
Dither a 24-bit RGB image to an 8-bit image using this colormap. The
optional \var{floyd} parameter can be set non-zero to use
Floyd-Steinberg error diffusion. This can be rather expensive in CPU
usage. 
\end{methoddesc}

\section{imgop ---
	Image Operations}
\declaremodule{extension}{imgop}
\modulesynopsis{Image Operations.}

This module contains some useful operations on images. It is expected
that it will eventually replace the \module{imageop} module, when it has
acquired all the needed functionality.

\begin{funcdesc}{shuffle}{data, width, height, srcfmt, dstfmt}
Shuffle bits around in pixels. The format of \var{data} is described
by \var{srcfmt}, the result is the image in format \var{dstfmt}. For
each pixel, \var{shuffle} extracts each component and inserts it in
the destination pixel. Components are truncated or extended in a
meaningful way when needed, so converting 8-bit RGB data to 24-bit
works as expected. This routine can also do stride-conversion, so it
can also be used to convert 4-byte aligned formats to 1-byte aligned
formats and vice versa.
\end{funcdesc}

\begin{funcdesc}{dither}{data, width, height, srcfmt, dstfmt}
Dither greyscale data to a bitmap using an order-8 ordered dither. The
source format can be either \var{grey} or \var{xgrey}, the destination
format must be \var{pbmbitmap}. Note that future implementations may
use a different dither algorithm, possibly selected with an optional
argument.
\end{funcdesc}

\begin{funcdesc}{unpack}{data, width, height, srcfmt, dstfmt}
Unpack data that packs multiple pixels in a byte. Source format should
have a single component only, and have an integral number of pixels in
a byte (so unpacking 3 5-bit pixels stored in a word will not work).
\end{funcdesc}

\section{imggif ---
	Support for GIF files.}
\declaremodule{extension}{imggif}
\modulesynopsis{Support for GIF files.}

This module supports reading and writing GIF files. It will read both
GIF87 and GIF89 files and writes GIF87 files unless transparency is
used. Many GIF features
are only partially supported, like non-square pixels (where you have
to do the conversion to square pixels yourself).
Reading interlaced files is supported, writing is not. Only
the first image from a GIF file can be read, and writing always
creates a GIF file with a single image.

GIF reader/writer objects define the following attributes, aside from
the ones described in the general imagefile reader/writer interface:

\begin{memberdesc}[imggifobject]{top}
Top offset of the image, as read from the GIF image header.
\end{memberdesc}

\begin{memberdesc}[imggifobject]{left}
Left offset of the image, as read from the GIF image header.
\end{memberdesc}

\begin{memberdesc}[imggifobject]{aspect}
The aspect ratio of the image. An aspect ratio of 0 or 49 means the
pixels are square.
\end{memberdesc}

\begin{memberdesc}[imggifobject]{transparent}
The colormap index of the color to be treated as transparent. For a
reader object, this attribute will not exist if transparency was not
used in the input file. Similarly, a writer will write a
non-transparent file unless this attribute is set.
\end{memberdesc}

\section{imgpbm ---
	Support for PBM files.}
\declaremodule{extension}{imgpbm}
\modulesynopsis{Support for PBM files.}

This module supports reading and writing PBMPLUS bitmap
images. Images are read from (and written to) PBM files.
Both the ASCII and the binary variants of the files are
supported.
PGM objects define the \var{width}, \var{height}, \var{format} and
\var{format_choices} attributes from the general interface. PBM
writers support one extra attribute:

\begin{memberdesc}[imgpbmwriter]{forceplain}
An integer flag. When set, the PBM file will be written in ASCII.
\end{memberdesc}

\section{imgpgm ---
	Support for PGM files.}
\declaremodule{extension}{imgpgm}
\modulesynopsis{Support for PGM files.}

This module supports reading and writing PBMPLUS greyscale
images. Images are read from either PGM or PBM files and written to
PGM files. Both the ASCII and the binary variants of the files are
supported.
PGM objects define the \var{width}, \var{height}, \var{format} and
\var{format_choices} attributes from the general interface. PGM
writers support one extra attribute:

\begin{memberdesc}[imgpgmwriter]{forceplain}
An integer flag. When set, the PGM file will be written in ASCII.
\end{memberdesc}


\section{imgppm ---
	Support for PPM files.}
\declaremodule{extension}{imgppm}
\modulesynopsis{Support for PPM files.}
This module supports reading and writing PBMPLUS color
images. Images are read from either PPM, PGM or PBM files and written to
PPM files. Both the ASCII and the binary variants of the files are
supported.
PPM objects define the \var{width}, \var{height}, \var{format} and
\var{format_choices} attributes from the general interface. PGM
writers support one extra attribute:

\begin{memberdesc}[imgppmwriter]{forceplain}
An integer flag. When set, the PPM file will be written in ASCII.
\end{memberdesc}

\section{imgpng ---
	Support for PNG files.}
\declaremodule{extension}{imgpng}
\modulesynopsis{Support for PNG files.}
This module supports reading and writing Portable Network Graphics
images. Images are read and written in their ``native'' format, with
some adaptations (greyscale pixels are extended to 8 bits on read, etc)

PPM objects define the \var{width}, \var{height}, \var{format},
\var{format_choices} and (for colormapped images) \var{colormap}
attributes from the general interface. 

\section{imgtiff ---
	Support for TIFF files.}
\declaremodule{extension}{imgtiff}
\modulesynopsis{Support for TIFF files.}

This module supports reading and writing of TIFF files. Currently
files with photometric tags of \code{minisblack}, \code{miniswhite}
and \code{rgb} are supported. Conversion of these formats from and to
incore image formats \code{rgb}, \code{rgb_b2t}, \code{grey} and
\code{grey_b2t} is handled. This is not according to the philosophy
sketched above and may change.

Many of the more obscure TIFF tag types are ignored upon input, so
reading files that contain them may not always produce the correct
images. TIFF files are always written with orientation top-left,
fillorder msb-to-lsb and LZW compression.

\section{imgjpeg ---
	Support for JPEG files.}
\declaremodule{extension}{imgjpeg}
\modulesynopsis{Support for JPEG files.}
This module supports reading and writing of JPEG files. formats
\code{rgb} and \code{grey} are supported. Of the zillions of options
that the IJG JPEG library supports one is accessible:

\begin{memberdesc}[imgjpegwriter]{quality}
In writer objects, this integer can be set to a value between 1 and
100 to denote at which quality level the jpeg file should be
coded. Low numbers give grainy images but small files. A quality below
25 results in a file that may not be portable.
\end{memberdesc}

{\em Note:}This module does not work if the SGI Compression Library is
configured into python (by including the python \var{cl} module). In
that case, it is replaced by a module written in Python providing the
same functionality by using the \var{cl} module.

\section{imgsgi ---
	Support for SGI RGB files.}
\declaremodule{extension}{imgsgi}
\modulesynopsis{Support for SGI RGB files.}

This module supports SGI image files. It supports \code{grey}, \code{xgrey},
\code{rgb8}, \code{xrgb8}, and \code{rgb} (both with and without alpha channel)
format files, and their \code{_b2t} equivalents.

Writer objects have two special attributes:

\begin{memberdesc}[imgsgiwriter]{rle}
An integer flag, defaulting to 1. When set, causes the file to be
written using RLE compression. When clear the data is written verbatim.
\end{memberdesc}

\begin{memberdesc}[imgsgiwriter]{rgba}
An integer flag, defaulting to 0. When set causes the alpha channel to
be saved to the file, otherwise only the RGB data is saved.  This has
only effect for \code{rgb} format files.
\end{memberdesc}

\begin{memberdesc}[imsgsgiobject]{name}
A string attribute.  This is the name of the image as found in the
image file itself.  When not set on writing, the name defaults to ``no
name''.  On reading, when there is no name in the image file, no
\code{name} attribute is created.
\end{memberdesc}

\section{imgpict ---
	Support for Macintosh PICT files.}
\declaremodule{extension}{imgpict}
\modulesynopsis{Support for Macintosh PICT files.}
This module supports writing Macintosh PICT files, from \code{macrgb}
images. It does not support reading PICT files yet.

\section{Credits}
A lot of the code in these modules was taken from other sources. All
code is freely distributable. See the source files for complete
copyright notices and attributions.

\section{To be done}
This section lists some of the things that need to be done, in no
particular order.

The tiff module should support colormapped tiff files, and maybe a few
other things as well.

Most modules are not yet coded efficiently. Also, they do not always
provide full support for the features of the image file formats
involved.
